\documentclass[11pt,addpoints,answers]{exam}

% Packages
\usepackage{fullpage}
\usepackage{latexsym,amssymb,amsfonts,amsmath,mathrsfs,float, amsthm}
\usepackage[font=small,labelfont=bf, width=.618\textwidth]{caption} % custom captions
\usepackage{xcolor}
\usepackage{tikz, graphics, enumerate}
\usepackage{hyperref}
\usepackage{algorithm} 
\usepackage{algpseudocode} 

% Paragraph indentation and spacing
\newcommand{\parset}{
	\setlength{\parskip}{3mm}
  	\setlength{\parindent}{0mm}}
	
% Probability
  \DeclareMathOperator{\var}{Var}
  \renewcommand{\Pr}{\mbox{\rm Pr}}	
  \newcommand{\Exp}{{\mathbb{E}}}
  \newcommand{\E}{\mathbb{E}}

% Sets 
  \newcommand{\R}{\mathbb{R}} % reals
  \newcommand{\C}{\mathbb{C}} % complex numbers
  \newcommand{\N}{\mathbb{N}} % natural numbers
  \newcommand{\Z}{\mathbb{Z}} % integers
  \newcommand{\F}{\mathbb{F}} % field
  \newcommand{\K}{\mathbb K} % field
  \newcommand{\T}{\mathbb T} % circle
\newcommand{\B}{\mathcal{B}} % ball
  \newcommand{\pmset}[1]{\{-1,1\}^{#1}} % hypercube in +-1 basis
  \newcommand{\bset}[1]{\{0,1\}^{#1}} % hypercube
  \newcommand{\sphere}[1]{S^{#1-1}} % real unit sphere of dimension #1
  \newcommand{\ball}[1]{B_{#1}} % real unit ball of dimension $1
  \newcommand{\Orth}[1]{O(\R^{#1})} % orthogonal group
  \DeclareMathOperator{\im}{im} % image
  \DeclareMathOperator{\vspan}{Span} % kernel
  \newcommand{\1}{\mathbf{1}}
  \DeclareMathOperator{\Cball}{\mathcal C}
  \DeclareMathOperator{\cay}{Cay} 
  \DeclareMathOperator{\sol}{Sol} 
\DeclareRobustCommand{\stirling}{\genfrac\{\}{0pt}{}}

% Miscellaneous
  \newcommand{\st}{:\,} % "such that" to define sets
  \newcommand{\ie}{{i.e.}}  
  \newcommand{\eg}{{e.g.}} 
  \newcommand{\eps}{\varepsilon}
  \newcommand{\ip}[1]{\langle #1 \rangle}
  \newcommand{\cF}{{\mathcal F}}   
  \DeclareMathOperator{\U}{\mathcal{U}}
  \DeclareMathOperator{\sign}{sign}
  \newcommand{\poly}{\mbox{\rm poly}}
  \newcommand{\ceil}[1]{\lceil{#1}\rceil}
  \newcommand{\floor}[1]{\lfloor{#1}\rfloor}
  \DeclareMathOperator{\Tr}{\mathsf{Tr}}
  \DeclareMathOperator{\diag}{diag}
  \DeclareMathOperator{\patt}{patt}
  \DeclareMathOperator{\argmin}{arg\,min}
  \DeclareMathOperator{\spec}{Spec}
  \DeclareMathOperator{\enc}{\sf Enc}
  \DeclareMathOperator{\dec}{\sf Dec}
  \DeclareMathOperator{\diam}{diam} 
  \DeclareMathOperator{\spn}{span} 
  \DeclareMathOperator{\geom}{gm}
\DeclareMathOperator{\sinc}{sinc}
\DeclareMathOperator{\disc}{disc}
\DeclareMathOperator*{\argmax}{arg\,max}
  \newcommand{\infnorm}{{\ell_\infty, \ldots, \ell_\infty}}
  \newcommand{\pnorm}{{\ell_p,\dots,\ell_p}}
  \newcommand{\tnorm}{{\ell_t,\dots,\ell_t}}
  
  
% Enviroments
  \newcommand{\beq}{\begin{equation}}
  \newcommand{\eeq}{\end{equation}}
  \newcommand{\beqn}{\begin{equation*}}
  \newcommand{\eeqn}{\end{equation*}}
  \newcommand{\beqr}{\begin{eqnarray}}
  \newcommand{\eeqr}{\end{eqnarray}}
  \newcommand{\beqrn}{\begin{eqnarray*}}
  \newcommand{\eeqrn}{\end{eqnarray*}}
  \newcommand{\bmline}{\begin{multline}}
  \newcommand{\emline}{\end{multline}}
  \newcommand{\bmlinen}{\begin{multline*}}
  \newcommand{\emlinen}{\end{multline*}}
  
  \makeatletter

% END OF SUPPLIED VARIABLES

  \chead{\large \textbf{Homework 1}}

  \lhead{\small
    \textbf{{CS 584/684}\\{PSU} \\ {Spring 2024}}}

  \rhead{\small \textbf{Due Date: April 17, 2024}\\ \textbf{Instructor:
     {Shravas Rao}}\\\textbf{Student}: FIRSTNAME LASTNAME}

  \setlength{\headheight}{20pt}
  \setlength{\headsep}{16pt}                       
  \headrule
% execute homework commands

\begin{document}

\pagestyle{head}                % put header on every page

\medskip
\noindent Answer the following questions.  You may collaborate with other members in your group.
However, you must write your own solution and include your sources.

\smallskip

\noindent \textbf{Collaborators}: WRITE YOUR COLLABORATORS HERE.

\begin{questions}

\question[6]

Consider the following sorting algorithm

\smallskip

\begin{algorithmic}
\State \textbf{Sort1}($A[1,2,\ldots,n]$)
\For {$i=n, n-1, \ldots, 1$}
\For {$j=n, n-1, \ldots, 2$}
\If {$A[j] < A[j-1]$}
\State swap $A[j]$ and $A[j-1]$
\EndIf
\EndFor
\EndFor
\end{algorithmic} 

\smallskip

 \begin{parts}

\part Consider an inductive proof of correctness of the above algorithm.
Give the base case and the statement proved in the inductive step (you do not need to give a proof).

  \begin{solution}
				
  \end{solution}

\part  Give the running time of the above sorting algorithm.

  \begin{solution}
				
  \end{solution}

Now, consider the following algorithm

\smallskip

\begin{algorithmic}
\State \textbf{Sort2}($A[1,2,\ldots,n]$)
\For {$i=n, n-1, \ldots, 2$}
\For {$j=n, n-1, \ldots, i$}
\If {$A[j] < A[j-1]$}
\State swap $A[j]$ and $A[j-1]$
\EndIf
\EndFor
\EndFor
\end{algorithmic} 

\smallskip

\part Give a counterexample for every $n \geq 3$ to prove that this sorting algorithm does not sort correctly.
Partial credit will be given for a counter example that works for a single $n$.

  \begin{solution}
				
  \end{solution}

\end{parts}

\question[4]  Give with proof, the running time of the following algorithms using big-Oh notation.

\begin{parts}
\part 

\smallskip

\begin{algorithmic}
\State \textbf{F1}($A[1,2,\ldots,n]$)
\State $s = 0$
\State $i = 1$
\While {$i \leq n$}
\State $j = 1$
\While {$j \leq n$}
\State $s = s+A[j]$
\State $j = j+1$
\EndWhile
\State $i = 2i$
\EndWhile
\end{algorithmic} 

\smallskip

  \begin{solution}
				
  \end{solution}
\part 

\smallskip

\begin{algorithmic}
\State \textbf{F1}($A[1,2,\ldots,n]$)
\State $s = 0$
\State $i = 1$
\While {$i \leq n$}
\State $j = 1$
\While {$j \leq n$}
\State $s = s+A[j]$
\State $j = j+i$
\EndWhile
\State $i = 2i$
\EndWhile
\end{algorithmic} 

\smallskip

  \begin{solution}
				
  \end{solution}
\end{parts}
  
  \question[5] Let $T(1) = 1$, and let
\[
T(n) = x^kT\left(\frac{n}{x}\right)+C n^k
\]
where $C$, $k$, and $x$ are constants.
Prove that
\[
T(n) = O(n^k \log(n))
\]
You may assume that $n$ is a power of $x$.
Hint: Use induction

\begin{solution}
  \end{solution}
    
\question[5] You are given an array $A[1, \ldots, n]$ of elements that can not be compared other than to say whether they are equal or not. This is, either $A[i] = A[j]$ or $A[i] \neq A[j]$. 
It is meaningless to say if $A[i] < A[j]$ (for example, if the elements of the array refer to images).

An element of $A$ appears with high majority if there are at least $3n/4$ copies in $A$.
Note that at most one element can appear with high majority.
Give an $O(n\log(n))$ algorithm to find the majority element of $A$ or determine if one does not exist.
You may assume that $A$ is a power of $2$.

Briefly explain why your strategy works and give a proof of the running time.

Hint: Use a divide and conquer strategy.

  \begin{solution}
  \end{solution}


\question[5] \textbf{Optional:} In class, we discussed the best-case and worst-case analysis of Insertion Sort. For this problem, we will consider the average-case analysis.

Consider the set of $n!$ permutations of a sorted list of length $n$, and assume that the list given is uniform over these permutations.
Give with proof, the expected value of the running time of insertion sort for this problem.

Hint: First, prove that the running time of insertion sort is equal to the number of inversions. Then give with proof, the expected value of the number of inversions.

  \begin{solution}
				
  \end{solution}

\end{questions}
\end{document}